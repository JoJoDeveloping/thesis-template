\makeatletter

% Packages needed
\usepackage[utf8]{inputenc}
\usepackage{geometry}
\usepackage[small,compact]{titlesec}
\usepackage[final]{listings}
\usepackage{amsmath}
\usepackage{amssymb}
\usepackage{tipa}
\usepackage[english]{babel}
\usepackage{lstautogobble}
\usepackage{proof}
\usepackage{bussproofs}
\usepackage{xparse}
\usepackage{needspace}
\usepackage{xspace}
\usepackage{mathpartir}
\usepackage{tikz}
\usetikzlibrary{arrows}
\usetikzlibrary{fit}
\usetikzlibrary{positioning}
\usepackage[amsmath,hyperref,thmmarks]{ntheorem}
\usepackage{parskip}
\usepackage{scrextend}
\usepackage{wrapfig}
\usepackage[nottoc]{tocbibind}
\usepackage{caption}

% For the font
\usepackage[T1]{fontenc}
\usepackage{tgpagella}
\usepackage[euler-digits,small]{eulervm}
\linespread{1.025}              % Palatino leads a little more leading

\usepackage{fancyhdr}
\usepackage[obeyDraft]{todonotes}
\setlength{\marginparwidth}{3cm}

\usepackage[toc]{glossaries}
\usepackage{natbib}
\bibpunct{[}{]}{,}{n}{,}{;}

\usepackage[breaklinks, draft=false, pdfusetitle]{hyperref}

%% Macros
\newcommand{\M}[1]{\ensuremath{\textit{#1}}} % \mathup
\newcommand{\imp}{\mathbin{\rightarrow}~}
\newcommand{\Imp}{\mathbin{\Rightarrow}~}
\renewcommand{\iff}{\mathbin{\leftrightarrow}}
\newcommand{\defeq}{\mathrel{\mathop:=}}

\newcommand{\dd}[2]{\ensuremath{#1_1,\dots,#1_{#2}}}  % x_1,...,x_n
\newcommand{\ddd}[2]{\ensuremath{#1_1\dots\,#1_{#2}}}  % x_1...x_n
\newcommand{\col}{\colon}
\newcommand{\set}[1]{\ensuremath{\{#1\}}}
\newcommand{\mset}[2]{\set{\,#1\mid#2\,}}
\newcommand{\eset}{\ensuremath{\emptyset}}
\newcommand{\incl}{\ensuremath{\subseteq}}
\newcommand{\bnfor}{~|~} %|
\newcommand{\qedsq}{\hfill\ensuremath{_\blacksquare}}

\newcommand{\blankpage}{\newpage
\thispagestyle{empty}
\mbox{}
\newpage}

\renewcommand\qed {{% set up
\parfillskip=0pt % so \par doesnt push \square to left
\widowpenalty=10000 % so we dont break the page before \square
\displaywidowpenalty=10000 % ditto
\finalhyphendemerits=0 % TeXbook exercise 14.32
%
% horizontal
\leavevmode % \nobreak means lines not pages
\unskip % remove previous space or glue
\nobreak % don't break lines
\hfil % ragged right if we spill over
\penalty50 % discouragement to do so
\hskip.2em % ensure some space
\null % anchor following \hfill
\hfill % push \square to right
$\square$% % the end-of-proof mark
%
% vertical
\par}} % build paragraph


\newlength{\blockindent}
\setlength{\blockindent}{8mm}

\newtheorem{lemma}{Lemma}[chapter]
\newtheorem{proposition}[lemma]{Proposition}
\newtheorem{fact}[lemma]{Fact}
%\newtheorem{theorem}[lemma]{Theorem} %Uncomment this line if you want to use usual theorems from ntheorem
\newtheorem{corollary}[lemma]{Corollary}
\newtheorem{definition}[lemma]{Definition}
\newtheorem{exercise}[lemma]{Exercise}


\newcommand{\setCoqFilename}[1]{\def\filename{#1}}
\def\filename{}

\newtheoremstyle{linkableTheorem}
  {\linkableTheoremAux{##1}{##2}}
  {\linkableTheoremAux{##1}{##2}[##3]}
\NewDocumentCommand\linkableTheoremAux{mmou\ignorespaces o}
 {\item[\hskip\labelsep \theorem@headerfont 
 \IfValueTF{#3}{\href{\baseurl\filename.html\##3}{#1\ #2\IfValueT{#5}{\ (#5)}{}}}{#1\ #2 \IfValueT{#5}{\ (#5)}{}}  
 \theorem@separator]#4\ignorespaces\IfValueT{#3}{\label{coq:#3}}}

\theoremstyle{linkableTheorem}
\newtheorem{theorem}[lemma]{Theorem} %Comment this line if you want to use usual theorems from ntheorem



%% Beweise
\theoremstyle{nonumberplain}
\theorembodyfont{\normalfont}
\theoremsymbol{\hfill \ensuremath{_\blacksquare}}
\newtheorem{proof}{Proof}
\providecommand*{\toclevel@proof}{0}
\theoremclass{LaTeX}


%% Geometrie
\newdimen\lines
\setbox\@tempboxa\hbox{(Programmierung)}
\lines=\ht\@tempboxa
\advance\lines by\dp\@tempboxa

\newlength{\exb}
\settowidth{\exb}{\normalfont x}
\geometry{%
  a4paper,%
  includehead,% (=> head is part of total body)
  ignorefoot,% (=> foot is not part of total body)
  top=3cm,% (top of paper |---| top of total body)
  totalwidth=70\exb,% (width of total body)
  totalheight=215mm,% (height of total body)
  headheight=1\lines,%
  headsep=2.5\lines,%
  foot=4\lines,% (bottom of text body |---| _bottom_ of foot)
  hcentering%
}%

\newcommand\coqref[1]{\ref{coq:#1}}

\newcommand{\setauthor}[1]{\author{#1}\def\theauthor{#1}}
\newcommand{\settitle}[1]{\title{#1}\def\thetitle{#1}}

\newcommand{\advisor}[1]{\def\theadvisor{#1}}
\newcommand{\reviewerOne}[1]{\def\thereviewerOne{#1}}
\newcommand{\reviewerTwo}[1]{\def\thereviewerTwo{#1}}
\newcommand{\university}[1]{\def\theuniversity{#1}}
\newcommand{\faculty}[1]{\def\thefaculty{#1}}
\newcommand{\thesistype}[1]{\def\thethesistype{#1}}
\newcommand{\subdate}[3]{\def\thesubday{#1} \def\thesubmonth{#2} \def\thesubyear{#3} }
\newcommand{\city}[1]{\def\thecity{#1}}
\newcommand{\coqUrl}[1]{\def\baseurl{#1}}

\newcommand{\maketitlepage}{
\begin{titlepage}
\begin{center}

\newcommand{\HRule}{\rule{\linewidth}{0.5mm}}

\textsc{{\Huge \theuniversity}}\\
\textsc{\thefaculty}\\[1.5cm]

\textsc{\thethesistype Thesis}\\[0.3cm]

% Title
\HRule \\[0.4cm]
{ \Huge \scshape \thetitle \\[0.4cm] }

\HRule \\[1.5cm]

% Author and supervisor
\noindent
\begin{minipage}{0.4\textwidth}
\begin{flushleft} \Large
\emph{Author}\\
\theauthor
\end{flushleft}
\end{minipage}%
\begin{minipage}{0.4\textwidth}
\begin{flushright} \Large
\emph{Advisor} \\
\theadvisor

\end{flushright}
\end{minipage}

\noindent
\begin{minipage}{1\textwidth}
\vspace{7cm}
\begin{center} \large
\emph{Reviewers} \\
\thereviewerOne\\
\thereviewerTwo
\end{center}
\end{minipage}

\vfill

{\large Submitted: \thesubday\textsuperscript{th} \thesubmonth~\thesubyear}

\end{center}
\end{titlepage}
}


\newcommand{\statementpage}{
~ \vspace*{2cm} ~

\textbf{Eidesstattliche Erkl\"arung}

Ich erkl\"are hiermit an Eides Statt, dass ich die vorliegende Arbeit
selbst\"andig verfasst und keine anderen als die angegebenen Quellen und
Hilfsmittel verwendet habe.

\textbf{Statement in Lieu of an Oath}

I hereby confirm that I have written this thesis on my own and that I
have not used any other media or materials than the ones referred to in
this thesis.


~ \vspace{3cm} ~

\textbf{Einverst\"andniserkl\"arung}

Ich bin damit einverstanden, dass meine (bestandene) Arbeit in beiden
Versionen in die Bibliothek der Informatik aufgenommen und damit
ver\"offentlicht wird.
% \begin{center}
% \Large

\textbf{Declaration of Consent}

I agree to make both versions of my thesis (with a passing grade)
accessible to the public by having them added to the library of the
Computer Science Department.

\vspace*{3cm}

\thecity, \thesubday$^{\text{th}}$ \thesubmonth, \thesubyear

\newpage
}

\makeatother
